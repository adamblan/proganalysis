\documentclass{mfcs}
\topic{\normalsize Adam Blank (adamblan@cs) and Michael Arntzenius (daekharel@gmail.com)}

\usepackage[backend=bibtex]{biblatex}
\bibliography{proposal}

\begin{document}

\begin{question}{Project Description}
  The high level goal of this project is to implement several functions. Given a
  Python program $P$,
\begin{itemize}
    \item \verb+ub_time_complexity(P)+ returns a symbolic upper bound on the \emph{run time} of $P$.
    \item \verb+ub_output_complexity(P)+ returns a symbolic upper bound on the \emph{size} of the output of $P$.
    \item \verb+lb_time_complexity(P)+ returns a symbolic lower bound on the \emph{run time} of $P$.
    \item \verb+lb_output_complexity(P)+ returns a symbolic lower bound on the \emph{size} of the output of $P$.
\end{itemize}

For example, consider the following code:
\begin{lstlisting}
def f(n):
    out = 1
    for i in range(n):
        out = out * n
    return out
\end{lstlisting}

We can see that \verb+ub_time_complexity(f)+ should return $n \in
\mathcal{O}(n)$, because the loop runs $n$ iterations. However, \verb+f+
\emph{returns} $\verb+out+$ which is $1$ initially and is multiplied by $n$, $n$
times. So, \verb+ub_output_complexity(f)+ $=n^n \in \mathcal{O}(n^n)$.

{\color{colour}\subsection{Restrictions}}

In order to keep the scope of this assignment reasonable, we will restrict input
programs in the following ways:
\begin{itemize}
\item All functions will be of type $\verb+int+, \verb+int+, ..., \verb+int+ \to
  \verb+int+$.
\item We will only simplify recursive programs of several particular forms
  (discussed below).
\item We will ignore certain dynamic features of Python.
\item If termination/invariant detection is too difficult to infer, we will
  output \verb+?+, meaning ``we do not know''.
\end{itemize}
\end{question}

The difficulty arises in handling loops and recursion.

{\color{colour}\subsection{Loops}}

% FIXME: cite paper
% CITING sas11-bound, pldi10_speed, cav09_speed
% Zuleger:2011:BAI:2041552.2041574
% Gulwani:2010:RP:1809028.1806630
% Gulwani:2009:SSC:1575060.1575069
To effectively bound the runtime of a loop, we must bound the number of
iterations. We propose doing this by using a variety of AI techniques and ideas
from \cite{Zuleger:2011:BAI:2041552.2041574, Gulwani:2010:RP:1809028.1806630,
  Gulwani:2009:SSC:1575060.1575069} to generate invariants on the values of
variables after multiple iterations. In particular, we aim to be able to handle
\emph{binary search}-like functions, and \emph{early terminations} of loops. We
intend to use search techniques on the AST of the loop, planning to prove the
invariants we find correct, and possibly machine learning to classify programs
as terminating or not.

{\color{colour}\subsection{Recursion}}

% TODO: clarify f(a) might not be called directly
To effectively bound the runtime of a recursive function, we need to be able to
bound the number of recursive calls. We only consider the class of programs
where one of the following is true, for $f(\textbf{x})$. We say $f(a_1, a_2,
\dots, a_n) \rightsquigarrow f(b_1, b_2, \dots, b_n)$, when $f(\textbf{b})$ is
called as part of executing $f(\textbf{a})$:
\begin{itemize}
\item $\exists i,m.\; \forall \textbf{y}.\; f(\textbf{x}) \rightsquigarrow
  f(\textbf{y}) \implies m \le y_i < x_i$
\item $\exists i,M.\; \forall \textbf{y}.\; f(\textbf{x}) \rightsquigarrow
  f(\textbf{y}) \implies x_i < y_i \le M$
\end{itemize}

As with loops, we will use planning and search to determine and prove whether
programs are of this form or not.


\begin{question}{References}
  \nocite{*}
  \printbibliography[heading=none]
\end{question}

\end{document}

% http://charlesneedham.com/en-us/um/people/sumitg/pubs/pldi10_speed.pdf
% http://www.chawdhary.co.uk/pubs/thesis.pdf
% ftp://ftp.cs.brown.edu/pub/techreports/90/cs90-25.pdf
% http://www.di.ens.fr/~cousot/publications.www/Cousot-81-PFA-ch10-p303-342.pdf
% http://www.cs.stonybrook.edu/~liu/papers/Alias-DLS10.pdf
% http://www.dcc.fc.up.pt/~nam/publica/artigoDYLA.pdf
% http://research.microsoft.com/en-us/um/people/sumitg/pubs/sas11-bound.pdf
% http://research.microsoft.com/en-us/um/people/sumitg/pubs/cav09_speed.pdf
